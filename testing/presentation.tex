\documentclass[11pt]{beamer}
\usetheme{default}
\usecolortheme{default}
\usepackage{amsmath,amssymb}
\usepackage{booktabs}
\usepackage{graphicx}

\setbeamertemplate{navigation symbols}{}
\setbeamertemplate{footline}[frame number]

\title{Extreme Value Theory for Influenza Peak Prediction}
\author{}
\date{}

\begin{document}

\frame{\titlepage}

\begin{frame}{Problem: Predicting Seasonal Flu Peaks}
\textbf{Research Question:} Can we predict the magnitude of seasonal influenza peaks across regions using extreme value theory?

\vspace{0.5cm}
\textbf{Challenge:}
\begin{itemize}
    \item Seasonal flu peaks vary widely (2-13\% ILI)
    \item Rare extreme seasons (e.g., 2017-18)
\end{itemize}

\vspace{0.5cm}
\textbf{Data:} CDC FluView ILINet
\begin{itemize}
    \item 14 seasons (2010-2024), 10 HHS regions
    \item 4,620 weekly observations
    \item 140 seasonal peak measurements
\end{itemize}
\end{frame}

\begin{frame}{Approach: Extreme Value Theory}
\textbf{Why EVT?} Specifically designed to model rare extreme events

\vspace{0.3cm}
\textbf{Generalized Extreme Value (GEV) Distribution:}
\[
F(x) = \exp\left\{-\left[1 + \xi\frac{x-\mu}{\sigma}\right]^{-1/\xi}\right\}
\]

\begin{itemize}
    \item $\mu$: location parameter (center of distribution)
    \item $\sigma > 0$: scale parameter (spread)
    \item $\xi$: shape parameter (tail behavior)
    \begin{itemize}
        \item $\xi > 0$: Heavy tail (rare extreme events possible)
        \item $\xi = 0$: Exponential tail (Gumbel distribution)
        \item $\xi < 0$: Bounded tail (upper limit exists)
    \end{itemize}
\end{itemize}

\vspace{0.3cm}
\textbf{Application:} Model distribution of seasonal peak maxima\\
\textbf{Training:} 80/20 split (112 train, 28 test region-seasons)
\end{frame}

\begin{frame}{Baseline: SIR Epidemiological Model}
\textbf{Susceptible-Infected-Recovered (SIR) Model:}
\[
\frac{dS}{dt} = -\beta \frac{SI}{N}, \quad \frac{dI}{dt} = \beta \frac{SI}{N} - \gamma I, \quad \frac{dR}{dt} = \gamma I
\]

\begin{itemize}
    \item $S$: susceptible population
    \item $I$: infected population
    \item $R$: recovered population
    \item $\beta$: transmission rate
    \item $\gamma$: recovery rate
    \item $R_0 = \beta/\gamma$: basic reproduction number
\end{itemize}

\vspace{0.3cm}
\textbf{Why SIR as baseline?}
\begin{itemize}
    \item Standard mechanistic model for epidemic dynamics
    \item Predicts peak infected counts based on transmission dynamics
    \item Tests whether mechanistic modeling can compete with statistical approach
\end{itemize}
\end{frame}

\begin{frame}{Evaluation Methods}
\textbf{Train/Test Split:}
\begin{itemize}
    \item Training: 112 region-season peaks (first 80\%)
    \item Testing: 28 region-season peaks (last 20\%)
\end{itemize}

\vspace{0.3cm}
\textbf{Model Fitting:}
\begin{itemize}
    \item GEV: Maximum likelihood estimation on training peaks
    \item SIR: Fitted separately for each test season's weekly data
\end{itemize}

\vspace{0.3cm}
\textbf{Prediction Task:}
\begin{itemize}
    \item GEV: Predict test peaks using 2-year return level
    \item SIR: Predict peak from fitted epidemic curve
\end{itemize}

\vspace{0.3cm}
\textbf{Performance Metrics:}
\begin{itemize}
    \item MAE: Mean Absolute Error
    \item RMSE: Root Mean Square Error
    \item MAPE: Mean Absolute Percentage Error
\end{itemize}
\end{frame}

\begin{frame}{Results}
\begin{table}
\centering
\begin{tabular}{lccc}
\toprule
Model & MAE & RMSE & MAPE (\%) \\
\midrule
\textbf{GEV}   & \textbf{2.19} & \textbf{2.72} & \textbf{29.4} \\
SIR   & 6.28 & 7.09 & 95.3 \\
\bottomrule
\end{tabular}
\end{table}

\vspace{0.3cm}
\textbf{Fitted GEV Parameters:}
\begin{itemize}
    \item $\mu = 3.70$, $\sigma = 1.92$, $\xi = 0.045$
    \item Goodness-of-fit: KS test $p = 0.888$
\end{itemize}

\vspace{0.2cm}
\textbf{Return Level Estimates:}
10-year: 8.2\% ILI \quad | \quad 100-year: 13.5\% ILI

\vspace{0.2cm}
{\small 2017-18 observed max (13.4\% ILI) aligns with 100-year estimate}
\end{frame}

\end{document}
